% Created 2024-07-21 Sun 16:58
% Intended LaTeX compiler: pdflatex
\documentclass[11pt]{article}
\usepackage[utf8]{inputenc}
\usepackage[T1]{fontenc}
\usepackage{graphicx}
\usepackage{longtable}
\usepackage{wrapfig}
\usepackage{rotating}
\usepackage[normalem]{ulem}
\usepackage{amsmath}
\usepackage{amssymb}
\usepackage{capt-of}
\usepackage{hyperref}
\usepackage[a4paper,left=1cm,right=1cm,top=1cm,bottom=1cm]{geometry}
\usepackage[american, english]{babel}
\usepackage{enumitem}
\usepackage{float}
\usepackage[sc]{mathpazo}
\linespread{1.05}
\renewcommand{\labelitemi}{$\rhd$}
\setlength\parindent{0pt}
\setlist[enumerate]{leftmargin=*}
\setlist[itemize]{leftmargin=*}
\setlist{nosep}
\date{}
\title{Why Worry About Research and Development}
\hypersetup{
 pdfauthor={Marcio Woitek},
 pdftitle={Why Worry About Research and Development},
 pdfkeywords={},
 pdfsubject={},
 pdfcreator={Emacs 29.4 (Org mode 9.8)}, 
 pdflang={English}}
\begin{document}

\thispagestyle{empty}
\pagestyle{empty}
\section*{Problem 1}
\label{sec:orgd5e2ced}

\textbf{Answer:}
\begin{itemize}
\item \ldots{}a product innovation.
\item \ldots{}a drastic innovation.
\end{itemize}
\section*{Problem 2}
\label{sec:org98c4baa}

\textbf{Answer:} Neither Garnier nor L'Oréal engage in R\&D\\

To answer this question, first we need to construct the payoff matrix. To do so,
we look at this situation from Garnier's perspective. The analysis for L'Oréal
is analogous. If only Garnier decides to invest in R\&D, then the possibilities
are as follows:
\begin{itemize}
\item the research is successful, and the firm earns an additional profit of 32;
\item the research is unsuccessful, and no additional profit is made.
\end{itemize}
Moreover, no matter the outcome, Garnier will have a cost of 10. Notice that
we're expressing amounts in units of 1 million Euros. Since the probability of
success is \(p=0.25\), the expected payoff for Garnier can be written as
\begin{equation*}
32\cdot p+0\cdot(1-p)-10=32p-10=-2.
\end{equation*}
Next, imagine that both firms decide to invest in R\&D. In this case, we have the
following possibilities:
\begin{enumerate}
\item Garnier is successful and L'Oréal isn't;
\item both companies are successful;
\item Garnier is unsuccessful.
\end{enumerate}
From Garnier's point of view, the corresponding values for the additional
profit are
\begin{enumerate}
\item 32;
\item 16;
\item 0.
\end{enumerate}
Don't forget about the fixed cost of 10. Then, in this situation, the expected
payoff for Garnier can be computed as follows:
\begin{align*}
  \begin{split}
    32\cdot p\cdot(1-p)+16\cdot p^2-10&=32p-32p^2+16p^2-10\\
    &=32p-16p^2-10\\
    &=32\cdot\frac{1}{4}-16\cdot\frac{1}{16}-10\\
    &=8-1-10\\
    &=-3.
  \end{split}
\end{align*}
To continue, we use the fact that Garnier and L'Oréal play symmetrical
roles. Consequently, the above calculations for Garnier are identical to
the ones we'd have to perform for L'Oréal. Then, at this point, we know
most entries of the payoff matrix. To write down this matrix, we need one
last fact: when both firms decide not to invest in R\&D, there's no loss or
additional profit for both of them. By combining all we know so far, we
finally obtain the payoff table for this game:
\begin{center}
\begin{tabular}{|c|c|c|}
\hline
Garnier $\backslash$ L'Oréal & No R\&D & R\&D\\
\hline
No R\&D & 0 / 0 & 0 / -2\\
R\&D & -2 / 0 & -3 / -3\\
\hline
\end{tabular}
\end{center}
By analyzing this table, it's possible to conclude that ``No R\&D'' is a
dominant strategy for both players. If we assume they're rational, then
this is the strategy they'll choose. Therefore, no company engages in R\&D.
\section*{Problem 3}
\label{sec:org034e1ef}

\textbf{Answer:}
\begin{itemize}
\item Monopolist with threat of an entrant
\end{itemize}
\section*{Problem 4}
\label{sec:orgf804de7}

\textbf{Answer:}
\begin{itemize}
\item \ldots{}are accused of hindering technological progress.
\item \ldots{}have been criticised by competition authorities in the past.
\item \ldots{}are used by companies to maintain a dominant position in their product
markets.
\item \ldots{}are used by companies to prevent competitors from inventing a close
rival to the own product.
\end{itemize}
\section*{Problem 5}
\label{sec:orgfb00b5e}

\textbf{Answer:}
\begin{itemize}
\item When it comes to decide about R\&D processes, firms often face a trade-off
between the chances of achieving innovations with promising returns on
the one side, and the risks of losing R\&D investments on the other side.
\item When undertaking a R\&D project, firms often face the threat that a
competitor comes up with the innovation first and absorbs the whole
market potential of the innovation.
\item Directly after implementing a drastic innovation, the innovator can
behave like a monopolist in the market.
\end{itemize}
\section*{Problem 6}
\label{sec:orgf49c7f5}

\textbf{Answer:} Yes\\

To answer this question, we need to compute the value of the innovation. We
begin by calculating the profit without the innovation. We denote this
amount by \(P_b\). By using the data given in the problem statement, we
can compute this profit as follows:
\begin{align*}
  \begin{split}
    P_b&=600\cdot(4-2)+400\cdot(3-2)\\
    &=600\cdot 2+400\cdot 1\\
    &=1200+400\\
    &=1600.
  \end{split}
\end{align*}
Then, before the innovation, the company earns a profit of 1600 Euros.\\
Next, consider the profit with the innovation, \(P_a\). If the
consultancy firm is correct, we'll have the following situation:
\begin{itemize}
\item 600 consumers are willing to pay 5 Euros per coffee capsule;
\item 400 consumers are willing to pay 4 Euros per coffee capsule.
\end{itemize}
Since the production cost is still 2 Euros per coffee capsule, the profit
\(P_a\) can be written as
\begin{align*}
  \begin{split}
    P_a&=600\cdot(5-2)+400\cdot(4-2)\\
    &=600\cdot 3+400\cdot 2\\
    &=1800+800\\
    &=2600.
  \end{split}
\end{align*}
This means the innovation will lead to a higher profit of 2600 Euros.\\
Finally, consider the value of the innovation. In this case, this amount
corresponds to the difference \(P_a-P_b=1000\) Euros. Since the
consultancy firm is charging half of this value for the innovation, it's a
good idea to accept the deal.
\section*{Problem 7}
\label{sec:org21b4185}

\textbf{Answer:} 249000\\

Since all companies have an equal share of the market, our firm sells
100,000 kg of paper. Moreover, every company charges the same price for
1,000 kg of paper. We denote this price by \(p\). Then, without the
innovation, our profit is
\begin{equation*}
P_b=100\cdot(p-500)=100p-50000.
\end{equation*}
Next, consider the situation after the innovation. In this case, we reduce
our price to \(p-1\). As a result, we take over the market, and start to
sell 1,000,000 kg of paper. Furthermore, our production cost becomes half
of what it was. More precisely, to produce a ton of paper, now we have to
spend 250 Euros. By combining all this information, we can determine the
profit with the innovation as follows:
\begin{equation*}
P_a=1000\cdot(p-1-250)=1000p-251000.
\end{equation*}
Hence, we can express the value of the innovation as
\begin{align*}
  \begin{split}
    V&=P_a-P_b\\
    &=1000p-251000-(100p-50000)\\
    &=1000p-251000-100p+50000\\
    &=900p-201000\\
    &=100(9p-2010).
  \end{split}
\end{align*}
To get a number from this expression, we need to know the value of the
price \(p\). We can calculate this price with the aid of the following
fact: originally, all companies are in perfect price competition.
Consequently, the profit \(P_b\) is zero: \(P_b=0\). By solving this
equation for \(p\), we get
\begin{align*}
  \begin{split}
    P_b&=0\\
    100p-50000&=0\\
    p-500&=0\\
    p&=500
  \end{split}
\end{align*}
Therefore, without the innovation, every company charges 500 Euros for a
ton of paper. With the aid of this result, we can finish the calculation of
\(V\):
\begin{align*}
  \begin{split}
    V&=100(9p-2010)\\
    &=100\cdot(9\cdot 500-2010)\\
    &=100\cdot(4500-2010)\\
    &=100\cdot 2490\\
    &=249000.
  \end{split}
\end{align*}
Hence the value of the innovation is 249,000 Euros.
\section*{Problem 8}
\label{sec:org7fd4aa6}

\textbf{Answer:} Efficiency Effect
\section*{Problem 9}
\label{sec:orgd4bd27c}

\textbf{Answer:} 9,900 Euros\\

First, consider the profit without the innovation. Originally, my company
sells tires to a fifth of all customers, i.e., 20 customers. However, for a
set of tires, the production cost is equal to the price. Consequently,
before the innovation, my profit is zero: \(P_b=0\).\\
Next, we analyze the situation after the innovation. In this case, the cost
to produce a set of tires is 100 Euros. Moreover, if I reduce my price to
199 Euros, my firm takes over the market and starts to sell tires to all
100 consumers. By combining all this information, we can compute the profit
with the innovation as follows:
\begin{equation*}
P_a=100\cdot(199-100)=9900.
\end{equation*}
Then the innovation leads to a higher profit of 9,900 Euros. Since
\(P_b=0\), the amount we've just obtained also corresponds to the value
of the innovation.
\section*{Problem 10}
\label{sec:org621f37a}

\textbf{Answer:}
\begin{itemize}
\item \ldots{}is subject to knowledge spillovers.
\end{itemize}
\end{document}
