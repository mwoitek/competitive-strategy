% Created 2024-07-14 Sun 13:07
% Intended LaTeX compiler: pdflatex
\documentclass[11pt]{article}
\usepackage[utf8]{inputenc}
\usepackage[T1]{fontenc}
\usepackage{graphicx}
\usepackage{longtable}
\usepackage{wrapfig}
\usepackage{rotating}
\usepackage[normalem]{ulem}
\usepackage{amsmath}
\usepackage{amssymb}
\usepackage{capt-of}
\usepackage{hyperref}
\usepackage[a4paper,left=1cm,right=1cm,top=1cm,bottom=1cm]{geometry}
\usepackage[american, english]{babel}
\usepackage{enumitem}
\usepackage{float}
\usepackage[sc]{mathpazo}
\linespread{1.05}
\renewcommand{\labelitemi}{$\rhd$}
\setlength\parindent{0pt}
\setlist[itemize]{leftmargin=*}
\setlist{nosep}
\date{}
\title{Why Firms Work Together}
\hypersetup{
 pdfauthor={Marcio Woitek},
 pdftitle={Why Firms Work Together},
 pdfkeywords={},
 pdfsubject={},
 pdfcreator={Emacs 29.4 (Org mode 9.8)}, 
 pdflang={English}}
\begin{document}

\thispagestyle{empty}
\pagestyle{empty}
\section*{Problem 1}
\label{sec:orged8b7a8}

\textbf{Answer:} Games with finite repetitions\\

This was explained in the lectures when the Street Lights example was discussed.
The basic idea is that, in the last repetition, no threat is credible. As a
result, players have an incentive not to cooperate. Then, by using backward
induction, it's possible to conclude that players will never cooperate. This is
the so-called endgame effect.
\section*{Problem 2}
\label{sec:orgd0aaf3c}

\textbf{Answer:}
\begin{itemize}
\item Soft commitment
\end{itemize}
\section*{Problem 3}
\label{sec:orgf7f6e06}

\textbf{Answer:} Boeing launches the 747X, Airbus does not launch the A380\\

Boeing plays aggressive commitment. This means Boeing commits to producing the
747X airplane. At this point, Airbus has two options:
\begin{itemize}
\item Airbus will produce the A380 airplane:
\begin{itemize}
\item Payoff for Boeing: -3;
\item Payoff for Airbus: -4.
\end{itemize}
\item Airbus will NOT produce the A380 airplane:
\begin{itemize}
\item Payoff for Boeing: 7;
\item Payoff for Airbus: 3.
\end{itemize}
\end{itemize}
In this case, using backward induction is trivial. After all, Boeing already
commited to a particular strategy. Consequently, there is only one subgame that
needs to be analyzed. Specifically, we only need to identify the best strategy
for Airbus. In order to maximize its payoff, Airbus should not produce the A380.
Therefore, the outcome of this game will be as follows:
\begin{itemize}
\item Boeing launches the 747X, and
\item Airbus does not launch the A380.
\end{itemize}
\section*{Problem 4}
\label{sec:orge254c9c}

\textbf{Answer:} No change\\

We begin by analyzing a single repetition of this game. Consider the best reply
by City Cuts for each strategy of Toby's Hairstyle:
\begin{itemize}
\item Toby's Hairstyle charges a high price: City Cuts charges a low price;
\item Toby's Hairstyle charges a low price: City Cuts charges a high price.
\end{itemize}
Next, consider the best reply by Toby's Hairstyle for each strategy of City
Cuts:
\begin{itemize}
\item City Cuts charges a high price: Toby's Hairstyle charges a high price;
\item City Cuts charges a low price: Toby's Hairstyle charges a high price.
\end{itemize}
Therefore, for Toby's Hairstyle, charging a high price is always the best
strategy. Moreover, for this company, it makes no difference if there's
cooperation or not. In both cases, if this player selects its best strategy,
then its payoff will be the same. This value corresponds to its maximum payoff.
So Toby's Hairstyle is free to choose its best option every time, since there
will be no repercussion. In other words, this player has no reason to deviate
from what's best for itself. As a consequence, its opponent has no reason to
respond differently. Hence the outcome of this game will not change.
\section*{Problem 5}
\label{sec:org302a47e}

\textbf{Answer:} Between 0.25 and 1\\

This game is very similar to the Diamond Cartel game. However, in the present
case, the payoff from cooperating is
\begin{equation}
P_c=\frac{120}{2}\frac{1}{1-p}=\frac{60}{1-p},
\end{equation}
where \(p\) denotes the probability that there will be a next season.
Moreover, the payoff from deviating is \(P_d=80\). The airlines will cooperate
when the payoff from cooperating is greater than the payoff from deviating:
\(P_c>P_d\). Hence:
\begin{align}
  \begin{split}
    P_c&>P_d\\
    \frac{60}{1-p}&>80\\
    \frac{3}{1-p}&>4\\
    \frac{1-p}{3}&<\frac{1}{4}\\
    1-p&<\frac{3}{4}\\
    p-1&>-\frac{3}{4}\\
    p&>1-\frac{3}{4}\\
    p&>\frac{1}{4}
  \end{split}
\end{align}
Since \(p\) represents a probability, the inequality \(p\leq 1\) must also
hold. Therefore, the correct answer is ``Between 0.25 and 1''.
\section*{Problem 6}
\label{sec:org6ba20f1}

\textbf{Answer:}
\begin{itemize}
\item High importance of future payoffs
\end{itemize}
\section*{Problem 7}
\label{sec:org68b879b}

\textbf{Answer:} \(M=120\) or greater\\

In this case, the probability that there will be a new game is \(p=0.5\).
Hence:
\begin{equation}
\frac{1}{1-p}=\frac{1}{1-\frac{1}{2}}=2.
\end{equation}
This result allows us to write the payoff from cooperating as follows:
\begin{equation}
\frac{M}{2}\frac{1}{1-p}=\frac{M}{2}\cdot 2=M.
\end{equation}
The companies will cooperate when this payoff is larger than the payoff from
deviating, which is 120. Therefore, the following inequality must be satisfied:
\(M>120\).
\section*{Problem 8}
\label{sec:org5052b50}

\textbf{Answer:} Situation with low interest rates\\

By analyzing the expression for the payoff from cooperating given in one of the
lectures, we conclude that this payoff tends to infinity as the interest rate
approaches zero. Equivalently, we can say that, when the interest rate is small,
the payoff from cooperating is large. As a result, in this case players are more
likely to cooperate.
\section*{Problem 9}
\label{sec:orgc7b6cdf}

\textbf{Answer:}
\begin{itemize}
\item It lowers competition
\item It makes future customers buy earlier
\end{itemize}
\section*{Problem 10}
\label{sec:org376d754}

\textbf{Answer:} No cooperation in first period - No cooperation in second period\\

By inspecting this game's payoff table, one can conclude that this is a
prisoner's dilemma. In this case, High Price corresponds to cooperation, and Low
Price corresponds to defection. So we're talking about a prisoner's dilemma with
a finite number of repetitions. We know that, in a situation like this, the
endgame effect will be observed. This means players will never cooperate.
Therefore, both burger stalls will choose to charge a low price every time.
\end{document}
